\documentclass[12pt]{article}
\usepackage{amsmath}
\usepackage{amssymb}
\usepackage{geometry}
\geometry{margin=1in}

\title{Mathematical Expression Formatting Guide}
\author{Mathematics Helper}
\date{\today}

\begin{document}

\maketitle

\section{Proper Fraction Formatting}

When writing mathematical expressions, especially fractions, proper formatting is essential for clarity. Here are examples of how to format fractions correctly:

\subsection{Original Problem}
Simplify: 
$$\frac{3^{-1} \times 9^{\frac{1}{2}}}{27^{-\frac{1}{3}}}$$

\subsection{Step-by-Step Solution}

\textbf{Step 1:} Rewrite each term using the same base where possible.
\begin{align}
3^{-1} &= \frac{1}{3} \\
9^{\frac{1}{2}} &= (3^2)^{\frac{1}{2}} = 3^1 = 3 \\
27^{-\frac{1}{3}} &= (3^3)^{-\frac{1}{3}} = 3^{-1} = \frac{1}{3}
\end{align}

\textbf{Step 2:} Substitute these values into the original expression:
$$\frac{3^{-1} \times 9^{\frac{1}{2}}}{27^{-\frac{1}{3}}} = \frac{\frac{1}{3} \times 3}{\frac{1}{3}}$$

\textbf{Step 3:} Simplify the numerator:
$$\frac{1}{3} \times 3 = 1$$

\textbf{Step 4:} Complete the division:
$$\frac{1}{\frac{1}{3}} = 1 \times \frac{3}{1} = 3$$

\subsection{Final Answer}
$$\frac{3^{-1} \times 9^{\frac{1}{2}}}{27^{-\frac{1}{3}}} = 3$$

\section{LaTeX Formatting Tips}

To properly format mathematical expressions:

\begin{enumerate}
    \item \textbf{Fractions:} Use \verb|\frac{numerator}{denominator}|
    \begin{itemize}
        \item Example: \verb|\frac{a}{b}| renders as $\frac{a}{b}$
    \end{itemize}
    
    \item \textbf{Exponents:} Use \verb|^{exponent}|
    \begin{itemize}
        \item Example: \verb|x^{2}| renders as $x^{2}$
    \end{itemize}
    
    \item \textbf{Mixed fractions in complex expressions:}
    \begin{itemize}
        \item Example: \verb|\frac{a + b}{c - d}| renders as $\frac{a + b}{c - d}$
    \end{itemize}
    
    \item \textbf{Nested fractions:}
    \begin{itemize}
        \item Example: \verb|\frac{\frac{a}{b}}{\frac{c}{d}}| renders as $\frac{\frac{a}{b}}{\frac{c}{d}}$
    \end{itemize}
    
    \item \textbf{For displayed equations (centered), use double dollar signs:}
    \begin{itemize}
        \item \verb|$$\frac{a}{b}$$| for:
        $$\frac{a}{b}$$
    \end{itemize}
\end{enumerate}

\section{Common Mathematical Formatting}

\begin{itemize}
    \item \textbf{Square roots:} \verb|\sqrt{expression}| $\rightarrow$ $\sqrt{expression}$
    \item \textbf{nth roots:} \verb|\sqrt[n]{expression}| $\rightarrow$ $\sqrt[n]{expression}$
    \item \textbf{Multiplication:} \verb|\times| $\rightarrow$ $\times$ or \verb|\cdot| $\rightarrow$ $\cdot$
    \item \textbf{Greek letters:} \verb|\alpha|, \verb|\beta|, \verb|\gamma| $\rightarrow$ $\alpha, \beta, \gamma$
    \item \textbf{Inequalities:} \verb|\leq|, \verb|\geq|, \verb|\neq| $\rightarrow$ $\leq, \geq, \neq$
\end{itemize}

\section{More Examples}

\begin{enumerate}
    \item \textbf{Complex fraction:}
    $$\frac{x^2 + 2x + 1}{x + 1} = \frac{(x+1)^2}{x+1} = x + 1$$
    
    \item \textbf{Fraction with exponents:}
    $$\frac{a^m}{a^n} = a^{m-n}$$
    
    \item \textbf{Multiple operations:}
    $$\frac{2^{-3} \times 4^{\frac{1}{2}}}{8^{-\frac{1}{3}}} = \frac{\frac{1}{8} \times 2}{\frac{1}{2}} = \frac{\frac{1}{4}}{\frac{1}{2}} = \frac{1}{2}$$
\end{enumerate}

\section{Best Practices for Mathematical Formatting}

\begin{enumerate}
    \item Always use proper LaTeX notation for mathematical expressions
    \item Use \verb|\frac{numerator}{denominator}| for all fractions
    \item For inline math, use single dollar signs: \verb|$math expression$|
    \item For displayed equations, use double dollar signs: \verb|$$math expression$$|
    \item Be consistent with notation throughout your document
\end{enumerate}

\end{document}