\documentclass[12pt]{article}
\usepackage{amsmath}
\usepackage{amssymb}
\usepackage{amsfonts}
\usepackage{mathtools}

\title{Mathematical Expressions Reference}
\author{Mathematical Formatting Guide}
\date{\today}

\begin{document}

\maketitle

\section{Fractions}

Basic fraction: $\frac{3}{4}$

Complex fraction: $\frac{3^{-1} \times 9^{\frac{1}{2}}}{27^{-\frac{1}{3}}}$

Continued fraction: $\frac{1}{2 + \frac{1}{3 + \frac{1}{4}}}$

\section{Exponentials}

Basic exponent: $x^2$

Exponential function: $e^{x}$

Fractional exponent: $x^{\frac{1}{2}}$

Negative exponent: $3^{-1}$

Complex exponent: $e^{x^2 + 2x + 1}$

\section{Roots}

Square root: $\sqrt{16}$

Cube root: $\sqrt[3]{27}$

Fourth root: $\sqrt[4]{x^2 + 1}$

\section{Logarithms}

Natural logarithm: $\ln(x)$

Log base 10: $\log(x)$

Log with base: $\log_2(x)$

General log: $\log_a(b)$

\section{Trigonometric Functions}

Basic functions: $\sin(x)$, $\cos(x)$, $\tan(x)$

Reciprocal functions: $\csc(x)$, $\sec(x)$, $\cot(x)$

Inverse functions: $\arcsin(x)$, $\arccos(x)$, $\arctan(x)$

Hyperbolic functions: $\sinh(x)$, $\cosh(x)$, $\tanh(x)$

Powers: $\sin^2(x)$, $\cos^n(x)$

\section{Matrices}

Basic 2x2 matrix:
\[
\begin{pmatrix}
a & b \\
c & d
\end{pmatrix}
\]

Determinant:
\[
\det\begin{vmatrix}
a & b \\
c & d
\end{vmatrix}
\]

\section{Summations and Integrals}

Summation: $\sum_{i=1}^{n} i$

Infinite series: $\sum_{k=0}^{\infty} \frac{x^k}{k!}$

Product: $\prod_{i=1}^{n} x_i$

Integral: $\int_a^b f(x) dx$

Double integral: $\iint_D f(x,y) dx dy$

Contour integral: $\oint_C \vec{F} \cdot d\vec{r}$

\section{Greek Letters}

Lowercase: $\alpha, \beta, \gamma, \delta, \epsilon, \zeta, \eta, \theta, \iota, \kappa, \lambda, \mu, \nu, \xi, \pi, \rho, \sigma, \tau, \upsilon, \phi, \chi, \psi, \omega$

Uppercase: $A, B, \Gamma, \Delta, E, Z, H, \Theta, I, K, \Lambda, M, N, \Xi, \Pi, P, \Sigma, T, \Upsilon, \Phi, X, \Psi, \Omega$

Special variants: $\varepsilon$ (variant epsilon), $\vartheta$ (variant theta), $\varpi$ (variant pi), $\varrho$ (variant rho), $\varsigma$ (variant sigma), $\varphi$ (variant phi)

\section{Inequalities}

$a < b$, $a > b$, $a \leq b$, $a \geq b$, $a \neq b$, $a \approx b$, $a \equiv b$, $a \propto b$

\section{Grouping and Parentheses}

Auto-sizing: $\left(\frac{a}{b}\right)$, $\left[\frac{a}{b}\right]$, $\left\{\frac{a}{b}\right\}$

Manual sizing: $\big( \Big( \bigg( \Bigg($ and their closing counterparts

\section{Complex Expressions}

Quadratic formula:
\[
x = \frac{-b \pm \sqrt{b^2 - 4ac}}{2a}
\]

Euler's formula:
\[
e^{i\pi} + 1 = 0
\]

Gaussian integral:
\[
\int_{-\infty}^{\infty} e^{-x^2} dx = \sqrt{\pi}
\]

Sine addition formula:
\[
\sin(\alpha + \beta) = \sin\alpha \cos\beta + \cos\alpha \sin\beta
\]

Piecewise function:
\[
|x| = \begin{cases}
x & \text{if } x \geq 0 \\
-x & \text{if } x < 0
\end{cases}
\]

\section{Your Example Expression}

Original expression: $\frac{3^{-1} \times 9^{\frac{1}{2}}}{27^{-\frac{1}{3}}}$

Let's simplify it step by step:

1. $3^{-1} = \frac{1}{3}$

2. $9^{\frac{1}{2}} = \sqrt{9} = 3$

3. $27^{-\frac{1}{3}} = \frac{1}{27^{\frac{1}{3}}} = \frac{1}{\sqrt[3]{27}} = \frac{1}{3}$

So the expression becomes: $\frac{\frac{1}{3} \times 3}{\frac{1}{3}} = \frac{1}{\frac{1}{3}} = 3$

Result: $\frac{3^{-1} \times 9^{\frac{1}{2}}}{27^{-\frac{1}{3}}} = 3$

\end{document}